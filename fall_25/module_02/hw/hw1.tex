\documentclass{article}

% Required packages
\usepackage{amssymb}
\usepackage{amsmath}
\usepackage{graphicx}
\usepackage{geometry}
\usepackage{tikz}
\usepackage{array}
\usepackage{booktabs}
\usepackage{enumitem}
\usepackage{listings}
\usepackage{xcolor}
\usepackage{fancyhdr}
\usepackage{float}
\usepackage{subcaption}
\usepackage{comment}

% Set page geometry
\geometry{a4paper, margin=1in}

% Configure listings for Python
\lstset{
  language=Python,
  basicstyle=\ttfamily\footnotesize,
  numbers=left,
  numberstyle=\tiny\color{gray},
  frame=single,
  breaklines=true,
  breakatwhitespace=true,
  captionpos=b,
  tabsize=4,
  showspaces=false,
  showstringspaces=false,
  showtabs=false,
  commentstyle=\color{gray}\textit,
  keywordstyle=\color{blue}\bfseries,
  stringstyle=\color{red}
}

\begin{document}

\pagestyle{fancy}
\chead{DSC 257: Unsupervised Learning (Fall 2025)}
\lhead{Homework 11}
\rhead{Randall Rogers}

%------------------
% Solution for 1(a) and 1(b)
%------------------
\subsection*{Solution 1}
\noindent\rule{\textwidth}{0.4pt}\\

\subsubsection*{Solution 1 (a)}
\subsubsection*{Step 1: Identify characteristics of the dataspace}
\parbox{\textwidth}{
We are given 10 dimensional vectors where each element can be any real number ($x_{i} \in \mathbb{R}$):
}


\subsubsection*{\normalfont}{$\therefore$ we can express the dataspace $\chi$ as: $\chi = \mathbb{R}^{10}$}

\noindent\rule{\textwidth}{0.4pt}\\

\subsubsection*{Solution 1 (b)}
\subsubsection*{Step 1: Identify characteristics of the dataspace}
\parbox{\textwidth}{
We are given 3 dimensional vectors where each element is zero or one ($x_{i} \in [0,1]$):
}

\subsubsection*{\normalfont}{$\therefore$ we can express the dataspace $\chi$ as: $\chi = {[0,1]}^{3}$}

\noindent\rule{\textwidth}{0.4pt}\\

\newpage

%------------------
% Solution for 2(a),2(b), and 2(c)
%------------------
\subsection*{Solution 2}
\noindent\rule{\textwidth}{0.4pt}\\
\subsubsection*{Solution 2 (a)}
\subsubsection*{Step 1: Define Euclidean distance ($\ell_2$)}
\parbox{\textwidth}{

$$\ell_2 = \|p - q\|_2 = \sqrt{\sum_{i=1}^{n} (p_i - q_i)^2}$$

}

\subsubsection*{Step 2: Compute $\ell_2$}
\parbox{\textwidth}{
Let $p=1$ and $q=10$
$$
\begin{aligned}
\ell_2 &= \sqrt{\sum_{i=1}^{n} (p_i - q_i)^2}\\
\ell_2 &= \sqrt{\sum_{i=1}^{1} (1 - 10)^2}\\
\ell_2 &= \sqrt{(- 9)^2}\\
\ell_2 &= 9
\end{aligned}
$$
}
\subsubsection*{\normalfont}{$\therefore$ $\ell_{2} = 9$}

\noindent\rule{\textwidth}{0.4pt}\\

\subsubsection*{Solution 2 (b)}
\subsubsection*{Step 1: Define Euclidean distance ($\ell_2$)}
\parbox{\textwidth}{

$$\ell_2 = \|p - q\|_2 = \sqrt{\sum_{i=1}^{n} (p_i - q_i)^2}$$

}

\subsubsection*{Step 2: Compute $\ell_2$}
\parbox{\textwidth}{
Let $p = \begin{bmatrix} -1 \\ 12 \end{bmatrix}, q = \begin{bmatrix} 6 \\ -12 \end{bmatrix}$
$$
\begin{aligned}
\ell_2 &= \sqrt{\sum_{i=1}^{n} (p_i - q_i)^2}\\
\ell_2 &= \sqrt{(p_1 - q_1)^{2}+(p_2 - q_2)^{2}}\\
\ell_2 &= \sqrt{(-1 - 6)^{2}+(12 - (-12))^{2}}\\
\ell_2 &= \sqrt{(-7)^{2}+(24)^{2}}\\
\ell_2 &= \sqrt{625}\\
\ell_2 &= 25
\end{aligned}
$$
}
\subsubsection*{\normalfont}{$\therefore$ $\ell_{2} = 25$}

\noindent\rule{\textwidth}{0.4pt}\\


\subsection*{Solution 2}
\noindent\rule{\textwidth}{0.4pt}\\
\subsubsection*{Solution 2 (c)}
\subsubsection*{Step 1: Define Euclidean distance ($\ell_2$)}
\parbox{\textwidth}{

$$\ell_2 = \|p - q\|_2 = \sqrt{\sum_{i=1}^{n} (p_i - q_i)^2}$$

}

\subsubsection*{Step 2: Compute $\ell_2$}
\parbox{\textwidth}{
Let $p = \begin{bmatrix} 1 \\ 5 \\ -1 \end{bmatrix}, q = \begin{bmatrix} 5 \\ 2 \\ 11 \end{bmatrix}$
$$
\begin{aligned}
\ell_2 &= \sqrt{\sum_{i=1}^{n} (p_i - q_i)^2}\\
\ell_2 &= \sqrt{(p_1 - q_1)^{2}+(p_2 - q_2)^{2}+(p_3 - q_3)^{2}}\\
\ell_2 &= \sqrt{(1 - 5)^{2}+(5 - 2)^{2}+(-1 - 11)^{2}}\\
\ell_2 &= \sqrt{(-4)^{2}+(3)^{2}+(-12)^{2}}\\
\ell_2 &= \sqrt{169}\\
\ell_2 &= 13
\end{aligned}
$$
}
\subsubsection*{\normalfont}{$\therefore$ $\ell_{2} = 13$}

\noindent\rule{\textwidth}{0.4pt}\\

\newpage

%------------------
% Solution for 3(a) and 3(b)
%------------------
\subsection*{Solution 3}
\noindent\rule{\textwidth}{0.4pt}\\
\subsection*{Solution 3 (a)}
\noindent\rule{\textwidth}{0.4pt}\\
\subsubsection*{Step 1: Normalize the vector $x$}
\parbox{\textwidth}{
Let $x = \begin{bmatrix} 10 \\ 15 \\ 25 \end{bmatrix}$

$$\sum_{i=1}^{3} x_{i} = x_{1} + x_{2} + x_{3} = 10 + 15 + 25 = 50$$

Now, divide each entry by the total sum:\\

$$p = \frac{1}{50} \cdot x = \frac{1}{50} \begin{bmatrix} 10 \\ 15 \\ 25 \end{bmatrix} = \begin{bmatrix} 10/50 \\ 15/50 \\ 25/50 \end{bmatrix} = \begin{bmatrix} 0.2 \\ 0.3 \\ 0.5 \end{bmatrix}$$
}
\subsubsection*{\normalfont}{$\therefore$ the result ($p$) of scaling vertor $x$ is the following:}
$$p = \begin{bmatrix} 0.2 \\ 0.3 \\ 0.5 \end{bmatrix}$$ \\

\subsection*{Solution 3 (b)}
\noindent\rule{\textwidth}{0.4pt}\\
\subsubsection*{Step 1: Define dimension of the probability simplex}
\parbox{\textwidth}{
The dimension of vector $p$ is $3$ and $k=n-1$ where $k$ is the dimension of the probability simplex
}
\subsubsection*{\normalfont}{$\therefore$ vector $p$ lies in the probability simplex($\Delta_2$) for $k=2$}

\noindent\rule{\textwidth}{0.4pt}\\

\newpage



\end{document}